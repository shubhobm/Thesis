\section{Data and model}
\label{sec:modelSection}

\subsection{The MCTFR data}
The familial GWAS dataset collected and studied by Minnesota Center for Twin and Family Research (MCTFR)\citep{LiEtal11, MillerEtal12, McGueEtal13} consists of samples from three longitudinal studies conducted by the MCTFR: (1) the Minnesota Twin Family Study (MTFS: \cite{IaconoEtal99}) that covers twins and their parent, (2) the Sibling Interaction and Behavior Study (SIBS: \cite{McGueEtal07}) that includes adopted and biological sibling pairs and their parents, and (3) the enrichment study (ES: \cite{KeyesEtal09}) that extended the MTFS by oversampling 11 year old twins who are highly likely to develop substance abuse. While 9827 individuals completed the initial assessments for participation in the study, after several steps of screening the final sample consisted of 7188 caucasian individuals clustered in 2300 nuclear families. %Here a total of 7188 Caucasian individuals, who come from $\sim 2300$ families, have been genotyped. A detailed description of the data is available at \cite{MillerEtal12}.

DNA samples collected from the subjects were analyzed using Illumina’s Human660W-Quad Array for 561,490 non-intensity SNP markers. After several data cleaning steps for quality control, 527,829 SNPs were retained. Covariates for each sample included age, sex, birth year, generation (parent or offspring), as well as two-way interactions between generation and other three covariates each. As for the quantitative phenotypes, five of them were studied in this GWAS: (1) Nicotine dependence, (2) Alcohol consumption, (3) Alcohol dependence, (4) Illegal drug usage, and (5) Behavioral disinhibition. The response variables corresponding to these phenotypes were derived from questionnaires using a hierarchical approach based on factor analysis \citep{HicksEtal11}.

\vspace{1em}
A more detailed description of the data is available  in \cite{MillerEtal12}. Several studies have been performed that focus on different aspects of this dataset. \cite{LiEtal11} used RFGLS to single out causal SNPs behind the height of participants, while \cite{McGueEtal13} used the same method to study SNPs behind the development of all five indicators of behavioral disinhibition mentioned above. \cite{IronsThesis12} focused on the effect of several factors affecting alcohol use in the study population, namely the effects of polymorphisms in the ALDH2 gene and the GABA system genes, as well as the effect of early exposure to alcohols as adolescents to adult outcomes. Finally \cite{CoombesThesis16} used a bootstrap-based combination test and a sequential score test to evaluate gene-environment interactions behind phenotypic outcomes in the data.

\subsection{Statistical model}
We shall use a Linear Mixed Model (LMM) with three variance components accounting for several potential sources of variation to model effect of SNPs behind a quantitative phenotype. This is known as \textit{ACE model} in the literature \citep{KohlerEtal11}. While the-state-of-the-art focuses on using a single SNP and other covariates as fixed effects, we shall incorporate all SNPs that are genotyped within a gene (or group of genes in some cases) in the set of fixed effects.

We assume that the families modeled are unrelated to one another, i.e. they are nuclear pedigrees. We stick to this structure for ease of representation, although the model fitting process remains unchanged for larger pedigrees. Suppose there are $m$ families in total, with the $i$-th pedigree containing $n_i$ individuals. Denote by $\bfy_i = (y_{i 1}, \ldots, y_{i n_i})^T $ the quantitative trait values for individuals in that pedigree, while the matrix $\bfG_i \in \BR^{ n_i \times p_g}$ contains their genotypes for a number of SNPs. Let $\bfC_i \in \BR^{ n_i \times p}$ denote the data on $p$ covariates for individuals in the pedigree $i$. Given these, we consider the following model.
%
\begin{align}\label{eqn:LMMeqn}
\bfY_i = \alpha + \bfG_i \bfbeta_g + \bfC_i \bfbeta_c + \bfepsilon_i
\end{align}
%
with $\alpha$ the intercept term, $\bfbeta_g$ and $\bfbeta_c$ fixed coefficient terms corresponding to the multiple SNPs and covariates, respectively, and $\bfepsilon_i \sim \cN_{n_i} ({\bf 0}, \bfV_i)$ the random error term. To account for the within-family dependency structure, we break up the random error variance into three independent components:
%
\begin{align}\label{eqn:partsOfV}
\bfV_i = \sigma_a^2 \bfPhi_i + \sigma_c^2 {\bf 1} {\bf 1}^T + \sigma_e^2 \bfI_{n_i}
\end{align}
%
The first component above represents a within-family random effect term to account for effects of other SNPs. The matrix $\bfPhi_i$ is the relationship matrix within the $i$-th pedigree. Its $(s,t)$-th element represents two times the kinship coefficient, which is the probability that two alleles, one randomly chosen from individual $s$ in pedigree $i$ and the other from individual $t$, are `identical by descent', i.e. come from same common ancestor. The second variance component accounts for shared environmental effect within the family, while the third term finally quantifies other sources of variation unique to an individual.

Following basic probability, the kinship coefficient of a parent-child pair is 1/4, a full sibling pair or non-identical (or dizygous = DZ) twins is 1/4, and for identical (or monozygous = MZ) twins is 1/2 in a nuclear pedigree. Following this, we can construct the $\bfPhi_i$ matrices for different types of families:
%
\begin{align*}
\bfPhi_{MZ} = \begin{bmatrix}
1 & 0 & 1/2 & 1/2 \\
0 & 1 & 1/2 & 1/2 \\
1/2 & 1/2 & 1 & 1\\
1/2 & 1/2 & 1 & 1
\end{bmatrix},
\bfPhi_{DZ} = \begin{bmatrix}
1 & 0 & 1/2 & 1/2 \\
0 & 1 & 1/2 & 1/2 \\
1/2 & 1/2 & 1 & 1/2\\
1/2 & 1/2 & 1/2 & 1
\end{bmatrix},
\bfPhi_{Adopted} = \bfI_4
\end{align*}
%
for families with parents (indices 1 and 2) and MZ twins, DZ twins, or two adapted children (indices 3 and 4), respectively.

We use the R package \texttt{regress} to fit the above model with additive error structure. The package requires specifying the dependency structure of all samples in the data. For ease of representation, we only consider nuclear pedigrees with MZ twins in our simulation study and data analysis, which simplifies the overall relationship matrix $\bfPhi = \text{diag} ( \bfPhi_1, \ldots, \bfPhi_m)$ as $\bfI_m \otimes \bfPhi_{MZ}$. Note that, situations in which the pedigree structure is not nuclear can be readily handled in this situation by supplying the overall $\bfPhi$ matrix. The second overall structural component will be  $\bfI_m \otimes {\bf 1} {\bf 1}^T$. The \texttt{regress} procedure includes the third structure in (\ref{eqn:partsOfV}) by default.