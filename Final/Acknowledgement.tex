I would like to thank my advisor, Prof. Snigdhansu Chatterjee for everything he has done for me. This dissertation would not have been possible without the very fundamental ideas behind it that came out of our discussions during the first two years of my PhD. Anshu da gave me the freedom to pursue those ideas and shape them as I wanted. I am thankful for the professional support he provided, as well as for always being available to listen to my ramblings. I am going to miss those weekly meetings in which we talked about anything and everything statistics.

I want to thank Prof. Saonli Basu of the Division of Biostatistics, with whom I collaborated during my last year. She has been an invaluable mentor, and discussions with her helped me learn about many applied aspects of the problems we had worked on. Thanks to her for being a reviewer of this dissertation as well. Special thanks to Profs. Lan Wang and Xiaotong Shen for reviewing the thesis and being on the final exam committee, and Prof. Gongjun Xu for being in my oral committee.

I would like to thank other professors in the department, who have always been helpful in answering questions, and the staff in the department office for their support regarding official matters. Thanks to other students in the department, for being there with feedback and discussions: especially Abhirup, Adam, Aaron, Daniel, Dootika and Sakshi.

I would like to acknowledge the National Science Foundation Climate Expeditions grant IIS-1029711, which provided me funding for 3 semesters, and the University of Minnesota Graduate School's Interdisciplinary Doctoral Fellowship (IDF), which supported me during my final year.

I consider myself very fortunate to have worked with a number of collaborators while still being a graduate student. Thanks to Profs. Matt McGue and Mike Miller of the Department of Psychology for their help during the IDF collaboration. Outside the campus, I am grateful to Prof. Subhash C. Basak of University of Minnesota Duluth for helping me write my first paper back in 2012 and all subsequent collaborations. Rayid Ghani of University of Chicago and Kush Varshney of IBM Research have been inspirations in developing my approach to collaborative research.

Coming to personal life, I would like to thank my friends for keeping me grounded and connected to the world outside. Special thanks to the three musketeers- Abhirup Mallik, Somnath Kundu and Suvankar Biswas, for being part of many happy memories during the past five years. I am grateful to Abhishek Nandy, for all his support during my first two years. Thanks to Amit da, Arja, Deepashree, Shriya, Tallin di, Taraswi di, Tushar and many more people associated with the Bengali Student Society of Minnesota, life in Minneapolis has been so much enjoyable.

I would like to thank my teachers from school, Prabhat Kusum Sarkar for introducing me to statistics, and Tapas Kumar Dhar for helping me build the necessary mathematical foundation. The past five years would not have been possible without the all-encompassing influence of Indian Statistical Institute (ISI), where I finished my bachelors and masters degree. This is too short a space to list all the avenues the ISI connection has helped me through in terms of personal and professional networking. I am thankful for being part of this ISI family.

Finally, I want to thank the people closest to me, my parents, for always being there no matter what. To them I owe more than anyone else in this world. In spite of coming from humble beginnings, they have the highest respect for the intellectual pursuit, and there is no substitute for the sacrifices they made in bringing me up. Ma and Baba, I dedicate this thesis to you.